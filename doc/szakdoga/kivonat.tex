
%----------------------------------------------------------------------------
% Abstract in hungarian
%----------------------------------------------------------------------------
\chapter*{Kivonat}\addcontentsline{toc}{chapter}{Kivonat}

Dolgozatomban kvantum ismétlők, valamint a velük kapcsolatos megoldandó feladatok, kihívások vizsgálatával, szimulációjával foglalkozok. \\
A kvantumos eszközök és eljárások számának növekedésével egyre nőnek a kvantumos információs rendszerek felé támasztott igények is. A jelenkori megoldások teljesítményének komoly gátat szab a távolság növekedése. Erre kínálhatnak megoldást az ``ún.'' kvantum ismétlők.\\
Figyelembe véve, hogy a választott téma egy viszonylag fiatal kutatási területhez tartozik, dolgozatomat egy rövid áttekintéssel, valamint a későbbiek megértéséhez szükséges általános bevezetővel nyitom. Ennek felhasználásával az összefonódás megosztás és az ezt használó ismétlő protokoll pontosabb elméleti bemutatásával, valamint a felmerülő feladatok, nehézségek vizsgálatával folytatom.  Ezután az általam készített szimuláció ismertetése következik. A dolgozatot az utolsó fejezetben már korábban említett problémák, feladatok szimulálásával, valamint a az így kapott eredmények értékelésével zárom.
\vfill

%----------------------------------------------------------------------------
% Abstract in english
%----------------------------------------------------------------------------
\chapter*{Abstract}\addcontentsline{toc}{chapter}{Abstract}
The purpose of my thesis is to describe the concept of the quantum repeater, further demonstrating its operation process and the associated challenges with a simulation.\\
With the growing number of quantum devices and protocolls, comes the need for better and better quantum communication systems. For such systems today, the distance between the endpoints still poses a huge limiting factor. One possible solution for this can be the use of quantum repeaters.\\
Considering the novelty of the investigated area, , I start my thesis with a brief historical overview , followed by a short introduction to the theoretical basics. I continue with describing the concept of entanglement swapping and exploring the possibility of a quantum repeater. Later I present the outline of my simulation. The thesis finishes with revisiting the challenges and problems related to the protocoll by reviewing the simulation results.
\vfill