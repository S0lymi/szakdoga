\clearpage
\begin{center}
\large
\textbf{Szakmai gyakorlat beszámoló}\\
Solymos Balázs - O60Z1K
\end{center}

\section*{Bevezető}

A szakmai gyakorlatot(Bsc villamosmérnök) idén nyáron végeztem a Qrypt Solutions kft. -nél. A szakmai konzulensem/mentorom Bacsárdi László volt, a gyakorlatom 2017 május végétől 6 hétig, 2017 július elejéig tartott. Azért választottam ezt a helyet a szakmai gyakorlatom elvégzésére, mert itt lehetőségem volt a szűkebb személyes érdeklődési körömbe is eső témával foglalkozni. Ez a téma a kvantumkommunikáció, melynek egyéb ágazataival valamelyest megismerkedtem korábban már az egyetemen az önálló laborom keretében,többek közt az ottani konzulensem ajánlotta ezt a lehetőséget, amivel később örömmel éltem. 

\section*{A gyakorlat}

A gyakorlat első hetében a szükséges adminisztráció elvégzése után, megkaptam az első elvégzendő feladatom, amivel az utána lévő hetekben foglalkoztam, valamint a hét végére egy belépő kártyát is a laborba, ahol a továbbiakban gyakorlatom végeztem, valamint kaptam egy asztalt erre a célra. Mivel lehetőségem volt a saját laptopom használatára a munka során ezért az asztalhoz biztosított eszközök közül csak egy plusz monitort használtam. Az itt töltött hat hetem feladatok szempontjából(később részletezve) két főbb részre osztható. Az első 3 hétben a kvantumkommunikáció helyzetét vizsgáltam Európában, különös figyelmet fordítva a kutatási támogatási rendszernek és végül a szerzett információkból készítettem egy rövidebb magyar nyelvű összefoglalót. A gyakorlat fennmaradó részében pedig kvantumkommunikáció fejlődését, helyzetét valamint a hozzá kapcsolódó fontosabb jelenségek bemutatását segítő anyagokat készítettem, melyek segítségével elkészült egy teljes plakátterv is. A feladatok egyéni feladatjaim voltak, ebből következik, hogy a gyakorlat alatt kizárólag önállóan dolgoztam, időnkénti konzultációval, valamint külső véleményezéssel megszakítva. Ezen tények miatt(saját laptop, önálló főleg irodalomkutatásból álló feladat) lehetőségem volt a laboron kívüli aktív munkavégzésre is, amit segített a flexibilisebb időbeosztás is, de hamar úgy találtam, hogy a labor által nyújtott környezet a legalkalmasabb számomra, különösen mikor a feladatokhoz szükséges információgyűjtéssel már végeztem.

\section*{Feladatok}

\subsection *{Quantum Manifesto}

A gyakorlat első felében feladatnak kaptam a közelmúltban közéttet Quantum Manifesto dokumentum, valamint az ehhez kapcsolódó Európai Uniós kutatási támogatási tervek, célkitűzések ismertetését, továbbá a kvantumkommunikáció európai helyzetének áttekintését. A cél ezen információk felhasználásával egy rövid 5-6 oldalas magyar nyelvű összefoglaló megteremtése volt. Elvárás volt még, hogy ez a dokumentum a LaTex nyelv segítségével írodjon, ezért ennek a nyelvnek az elsajátítása is a feladat részét képezte.Az össefoglaló elkésztéséhez szükséges idő nagy része azonban irodalomkutatással telt, amit talán a kutatott terület fiatalságának valamint a rendelkezésre álló EU-s dokumentumok mennyisége indokol. Az ehhez a feladathoz felhasznált források többsége megtalálható a QUROPE honlapján: \url{http://qurope.eu/}. Az itt feldolgozott témát talán a maga a készített dokumentum írja le, összefoglaló jellegének köszönhetően, úgyhogy a továbbiakban a munka menetete szempontjából fontos, viszont az összefoglalóban nem említett dolgokat foglalom össze. Maga a Quantum Manifesto dokumentum csak 20 oldal hosszú, de mivel sok más programmal együtt az EU H2020-as tervének a része, ezért a pontosabb megértéshez és elemzéshez szükséges annak az ismerete. Szerencsére ehhez is létezik egy 150 oldalas ``QT roadmap'' ami a kvantuminformatikai célkitűzéseket foglalja pontosabban össze, amit az összefoglaló készítése során fel is használtam. Továbbá, hogy az uniós általánosabb kvantumkommunikációs helyzetről pontosabb információkhoz juthassunk, magával az uniós támogatási rendszerrel, pontosabban ennek nyílvántartásával is érdemes jobban megismerkedni. Ez fontos, mivel maga a kvantumkommunikáció egy elég szűk kutatási területet képvisel, ezért ha a hozzá kapcsolható vállalatokat, csoportokat szeretnénk felkutatni ellenőrizni kell, hogy ami támogatást kaptak egy általában átfogóbb kvantumos témában, pontosan milyen projektre lett megadva. Ennek segítségével ki tudjuk szűrni a nagyobb halmazból a számunkra érdekes kisebbet.
A jelenlegi helyzet felméréséhez hasznos források volt még továbbá a  \url{http://qurope.eu/} oldalon található adatbázisok az ipari partnereket tekintve. Természetesen itt is pontosan ellenőrizni kellett a témábavágóságot. Az elkészült dokumentum a következő:


